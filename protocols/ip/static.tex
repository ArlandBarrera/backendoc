A static IP address is a manually configured signifier assigned to a device that remains consistent and cannot change across multiple networks sessions. Individuals do not typically need a static IP address, but businesses need them to host their own servers.

An IP address is considered to be static if the same IP address is assigned every time the user or device connects. Static IP addresses are particularly useful for enterprises that need to guarantee server and websites uptime. The also offer reliable internet connections, quicker data exchanges, and more convenient remote access.

\textbf{Advantages}

\begin{itemize}
  \item \textbf{Better online name resolution:} devices with static IP addresses can be reliably discovered and reached via thier assigned hostnames and do not need to be tracked for changes. For this reason, components like \textit{File Transfer Protocol} (\textbf{FTP}) servers and web servers use fixed addresses.
  \item \textbf{Anywhere, anytime access:} makes a device accessible anywhere in the world. Can make it quick for and easy for people to locate and use shared devices, such as a printer on their network.
  \item \textbf{Reduced connection lapses:} connection lapses typically happens when devices are not recognized by a network. An IP address that never resets or adjusts is essential for device processing vast amounts of data.
  \item \textbf{Faster download and upload speed:} devices with static IP addresses enjoy higher access speeds, which is essential for heavy data users.
  \item \textbf{Accurate geolocation data:} provides access to precise geolocation data. More accurate data means businesses are better able to manage and log incidents in real time, as well as detect and remediate potential attacks before they cause damage to networks.
\end{itemize}

\textbf{Disadvantages}

\begin{itemize}
  \item \textbf{Easy-to-track addresses:} the constant nature of statis IP addresses makes it easy for people to track a device the data they access or share. This could be a securiry concern, giving cyber criminals a route into a machine and subsequently unauthorized access to corporate networks.
  \item \textbf{Post-breach difficulties:} static IP addresses increase the risk of a website being hacked. In the aftermath of a data breach, they also make it more difficult to change IP addresses, making the business more susceptible to ongoing issues.
  \item \textbf{Cost issues:} the cost is significantly high. Many Internet service and hosting providers require users to sign up for commercial accounts or pay one-time fees in order to assign a static IP on each of their devices and websites.
\end{itemize}
