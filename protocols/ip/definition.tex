IP stands for \emph{Internet Protocol}, a set of rules and a unique numerical label, called and an IP Adress, assigned to very device on a network, such as the internet. IP addresses allow commputers and other devices to send and receive data by identifying their specific location and ensuring that data packets reach the correct destination, enabling communication between different networks.

An IP address helps devices to find whatever data or content is located to allow for retrieval. Commom tasks for an IP address include both the identification of a host or a network, or identifying the location of a device. An IP address is not random, it's creation has the basis of math. With the mathematical assignment of an IP address, the unique identification to make a connection to a destination can be made.

\begin{enumerate}
  \item \textbf{Data packets:} Data sent over the internet is broken down into smaller pieces called packets.
  \item \textbf{IP information:} Each packet is given and IP address, acting as the "electronic return address" for these packets.
  \item \textbf{Routing:} Routers read this IP information and direct the packets to the right place, allowing machines to communicate with each other even if they are on different networks.
\end{enumerate}

IP addresses can be classified by:

\begin{table}[H]
  \centering
  \begin{tabular}{|c|c|}
    \hline
    \textbf{Classification method} & \textbf{Types} \\
    \hline
    Version or standards & IPv4, IPv6 \\
    \hline
    Function & Public, Private \\
    \hline
    Assignment & Static, Dynamic \\
    \hline
  \end{tabular}
\end{table}
