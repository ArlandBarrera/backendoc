Used for devices within a LAN (like a home or an office), is not accessible from the Internet, and is assigned by a router. This facilitates communication between devices on a private network, offering increased security by limiting external visibility. Examples are a computer's or smat TV's IP address when connected to a home Wi-Fi.

By using private IP addresses internally, organizations can avoid the need to obtain and manage large blocks of public IP addresses, reducing costs associated with Internet connectivity. Requires \textit{Network Address Translation} (\textbf{NAT}), which can cause overhead in terms of processing energy, latency, and complexity, in particular in big-scale deployments.

The range of private IP addresses are:

\begin{itemize}
  \item \textbf{Class A:} 10.0.0.0 - 10.255.255.255, for large networks wich many devices.
  \item \textbf{Class B:} 172.16.0.0 - 172.31.255.255, for medium-sized networks, such as schools or businesses.
  \item \textbf{Class C:} 192.168.0.0 - 192.168.255.255, for small networks, such as homes or small offices.
\end{itemize}
