\textit{Internet Service Providers} (\textbf{ISPs}) temporarily assign dynamic IP addresses via the \textit{Dynamic Host Configuration Protocol} (\textbf{DHCP}) server. This means an IP address can change every time a user reboots their router or system, and when the user connects to their ISP server. When not in use, a dynamic IP address can be automatically assigned to another device. This makes dynamic IP addresses more suitable for home networks than large organizations.

An IP address is considered to be dynamic if it is pulled from a pool of available IP addresses bby the router every time the user or device connects to the network. Dynamic IP addresses are better suited for home networks and personal Internet use.

\textbf{Advantages}

\begin{itemize}
  \item \textbf{Cost reduction:} obtaining a dynamic IP address is typically automated, making it a more cost-efficient option.
  \item \textbf{Enhance security:} devices with dynamic IP addresses are more difficult to track, reducing the risk of attackers targeting business networks.
  \item \textbf{Improved configuration:} dynamic IP addresses are automatically configured by a DHCP server, which removes the need for users to do so manually.
  \item \textbf{Increased flexibility:} different devices can reuse addresses and are assigned a new IP address every time they join a network.
\end{itemize}

\textbf{Disadvantages}

\begin{itemize}
  \item \textbf{Hosting problems:} the changing nature of dynamic IPs means users may encounter problems with the \textit{Domain Name System} (\textbf{DNS}). This makes dynamic IP addresses less effective for hosting servers and websites and tracking geolocations.
  \item \textbf{Poor technical reliability:} dynamic IP addresses can result in frequent periods of downtime and connection dropout issues, which makes them ineffective for data-intense online activities.
  \item \textbf{Remote access} users with dynamic IP addresses may have trouble accessing the Internet from devices outside their primary network.
\end{itemize}
