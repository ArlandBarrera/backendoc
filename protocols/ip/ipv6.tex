IPv4 has not been able to cope with the massive explosion in the quantity and range of devices beyond simply mobile phones, desktop computers and laptops. The original IP address format was not able to handle the number of IP addresses being created.

To address this problem, IPv6 was introduced. This new standard operates a hexadecimal format, that means billions of unique IP addresses can now be created. As a result, the IPv4 system that could support up to 4.3 billion unique numbers has been replaced by an alternative that, theoretically, offers unlimited IP addresses.

This is because an IPv6 IP address consists of eight groups that contain four hexadecimal digits, which use 16 distinct symbols of 0 to 9 followed by A to F to represent values of 10 to 15. It uses a \textbf{128-bit} alphanumeric address written using a colon-separated hexadecimal notation, for example \textbf{2001:0db8:85a3:0000:0000:8a2e:0370:7334}. IPv6 also has a \textbf{shorthand notation}, for example \textbf{2001:db8:85a3::8a2e:370:7334}, this is typically used for convenience and to avoid visual clutter.

With 128-bit address space, it allows \textbf{340 undecillion} unique address space. IPv6 supports a theoretical maximum of \textbf{340 282 366 920 938 463 463 374 607 431 768 211 456}.

IPv6 is not inherently faster than IPv4 in terms of raw speed. However, it improves the efficiency of data transmission. Despite the differences, both protocols can run concurrently, allowing for seamless transitions in compatibility ans support across various network infrastructures. Many modern network devices, including routers and switches, are equipped to handle both protocols.

The main reason for the little adoption of IPv6 is cost of maintenance. Manage a system that uses IPv4 and instead use IPv6 is very demanding. Though the adoption is slow and steady.
