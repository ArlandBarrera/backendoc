IPv4 has not been able to cope with the massive explosion in the quantity and range of devices beyond simply mobile phones, desktop computers and laptops. The original IP address format was not able to handle the number of IP addresses being created.

To address this problem, IPv6 was introduced. This new standard operates a hexadecimal format, that means billions of unique IP addresses can now be created. As a result, the IPv4 system that could support up to 4.3 billion unique numbers has been replaced by an alternative that, theoretically, offers unlimited IP addresses.

This is because an IPv6 IP address consists of eight groups that contain four hexadecimal digits, which use 16 distinct symbols of 0 to 9 followed by A to F to represent values of 10 to 15.
