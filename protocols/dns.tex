The \textit{Domain Name System} (\textbf{DNS}) is a hierarchical and distributed naming system that translates domain names into IP addresses. Humans access information online through \textbf{domain names} and web browsers interact through IP addresses. When a user types a domain like \textbf{www.example.com} into a browser, DNS ensures that the request reaches the correct server by resolving the domain to it's corresponding IP address. Without DNS, users would have to remember the numerical IP address of every website we want to visit, which is highly impractical. It is the \textbf{phonebook of the internet}.

The process of \textbf{DNS Lookup}, also called \textbf{DNS resolution}, involves converting a hostname (such as www.example.com) into a computer-friendly IP address, which computers use to locate and communicate with each other on the internet.

\textbf{Reverse DNS Lookup} is the proccess of mapping an IP address back into it's corresponding domain name. This is used for network and email security.
