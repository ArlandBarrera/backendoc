Depth First Search (DFS) is a graph traversal method that starts from a spurce vertex and explores each path completely before backtracking and exploring other paths. To avoid revisiting nodes in graphs in cycles, a visited array is used to track visited vertices. The data structure that keeps track of the visited vertices is a stack.

\subsubsection*{Steps}

\begin{itemize}
  \item Start from a given source vertex and explore one path as deeply as possible.
  \item When it reaches a vertex with no unvisited neighbors, it backtracks to the previous vertex to explore other unvisited paths.
  \item This continues until all reachable vertices from the start are visited.
\end{itemize}

\subsubsection*{Implementation}

This implementation is for a connected and undirected graph. All vertices are reachable in the graph and the direction of the edges go both ways.

\begin{lstlisting}[style=general]
function dfsRec(adj, visited, src, res) {
  visited[src] = true;
  res.push(src);

  // recursively visit all adjacent vertices
  for (int i of adj[src]) {
    if (!visited[i]) {
      dfsRec(adj, visited, i, res);
    }
  }
}

function dfs(adj) {
  int visited = new Array(adj.length).fill(false);
  int res = [];
  dfsRec(adj, visited, 0, res);
  return res;
}

function addEdge(adj, u, v) {
  adj[u].push(v);
  adj[v].push(u);
}

int vertices = [[1,2], [1,0], [2,0]];
int V = nodes.length;
int adj = [];

for (int i = 0; i < V; i++) {
  adj.push([]);
}

for (int i = 0; i < vertices.length; i++) {
  addEdge(adj, vertices[i][0], vertices[i][1]);
}

int res = dfs(adj);
\end{lstlisting}

\subsubsection*{Time and Space Complexity}

\textbf{Time Complexity}

\textbf{Time Complexity:} \textit{O(V + E)}, where V is the number of vertives and E is the number of edges in the graph.

\textbf{Space Complexity:} \textit{O(V + E)}, since an extra visited array of size V is required, and stack size for recursive calls.

\subsubsection*{Advantages and Disadvantages}

\textbf{Advantages:}

\begin{itemize}
  \item x
\end{itemize}

\textbf{Disadvantages:}

\begin{itemize}
  \item x
\end{itemize}
