Search for maximum and minimum element in an array or list using \linearsch.

\subsubsection*{Steps}

\begin{itemize}
  \item Traverse the array from the first index until the last comparing the current value with the next, and store the maximum value in a variable.
\end{itemize}

\subsubsection*{Implementation}

\begin{lstlisting}[style=general]
// function for maximum value
function max(arr, n) {
  int max = 0;
  for (int i = 0; i < n; i++)
    if (arr[i] > max)
      max = arr[i];
  return max;
}

// function for minimum value
function min(arr, n) {
  int min = 0;
  for (int i = 0; i < n; i++)
    if (arr[i] < min)
      min = arr[i];
  return min;
}

// main
int arr = [1, 56, -23, 101, 34, 6, -11];
int n = arr.length;
// max
max(arr, n);
// min
min(arr, n);
\end{lstlisting}

\subsubsection*{Time and Space Complexity}

\textbf{Time Complexity:} \textit{O(n)}.

\textbf{Space Complexity:} \textit{O(1)}.

\subsubsection*{Advantages and Disadvantages}

\textbf{Advantages:}

\begin{itemize}
  \item Best for small arrays.
\end{itemize}

\textbf{Disadvantages:}

\begin{itemize}
  \item Inefficient for big arrays.
\end{itemize}
