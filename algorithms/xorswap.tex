Given two values \textbf{a} and \textbf{b}, they can be swaped without the need of an temporary variable using the \textbf{xor}, exclusive or, operator. This works by changing the bits of the values. The caret symbol `\verb|^|' is the most common operator for the XOR operation in many programming languages like c, c++, Java and Javascript .The process is the following:

\begin{lstlisting}[style=general,caption=XOR swap]
  Algorithm swapValuesXOR(a, b){
    a = a ^ b
    b = a ^ b
    a = a ^ b
  }
\end{lstlisting}

XOR only returns true (\emph{1}) if the compared values are in an \emph{or} state, otherwise returns false (\emph{0}).

There are three steps that involve the operation between \textbf{a} and \textbf{b} using \textbf{xor}. In the first step the result is stored in $a$, then in $b$ in the second and lastly in the third in $a$ again.

This method is used in low level languages such as assembly.

\textbf{Example:}

a = 5 and b = 7, in binary a = 101 and b = 111.

First step, the result is stored in $a$:

\begin{align*}
  a &= 101 \\
  b &= 111
\end{align*}

\[\boxed{
  a = 010
}\]

Second step, the result is stored in $b$:

\begin{align*}
  a &= 010 \\
  b &= 111
\end{align*}

\[\boxed{
  b = 101
}\]

Third step, the result is stored in $a$ again:

\begin{align*}
  a &= 010 \\
  b &= 101
\end{align*}

\[\boxed{
  a = 111
}\]

a = 111 and b = 101, in other terms a = 7 and b = 5. The values have been swaped.
