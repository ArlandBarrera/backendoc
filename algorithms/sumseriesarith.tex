The sum of an arithmetic series is is given by the expression:

\[\boxed{
  S_n = \dfrac{n}{2}\left(a_i+a_f\right)
}\]

Computers process multiplications faster, so the division $n/2$ can be replaced by the product $n*0.5$.

\textbf{Elements:}

\begin{itemize}
  \item $S_n$ = sum of series.
  \item $n$ = number of terms.
  \item $a_i$ = initial term.
  \item $a_f$ = final term.
\end{itemize}

To find $n$, the formula for the $n^{th}$ term of an arithmetic progression can be used:

\[\boxed{
  a_f = a_i + (n-1)d
}\]

Where $d$ is the step, the standard difference between each term.

By isolating $n$, the formula ends up like this:

\[\boxed{
  n = \left(\dfrac{a_f-a_i}{d}\right)+1
}\]

The algorithm has the following structure:

\begin{lstlisting}[style=general]
// if n is unknown
n = ((af - ai) /d) + 1

Algorithm sumArithmeticSeries(n, ai, af){
  s = n * 0.5 * (ai + af)
}
\end{lstlisting}

This algorithm is $O(1)$, constant.

\textbf{Example:}

The sum of odd numbers (1, 3, 5, 7, ...) between 1 and 100. The range of odd values is 1-99, the first term is 1 and the last is 99. The step between eaxh term is 2, with $n$ can be found.

\begin{align*}
  n &= \left(\dfrac{99-1}{2}\right)+1 \\
  n &= \left(\dfrac{98}{2}\right)+1 \\
  n &= 49+1 \\
  n &= 50
\end{align*}

The sum of the arithmetic series with $n$ = 50 is:

\begin{align*}
  S_{50} &= 50*0.5\left(1+99\right) \\
  S_{50} &= 25*100 \\
  S_{50} &= 2500
\end{align*}
